\documentclass[11 pt]{article}
\usepackage[english]{babel}
\usepackage{amsmath,amssymb,amsthm}

\title{Problem set 1}
\author{Pablo Boixeda Alvarez}

\newtheorem{Prob}{Problem}
\newtheorem{Prop}{Proposition}
\newtheorem{Opt}{Optional Problem}

\theoremstyle{definition}
\newtheorem{Not}{Notation}
\newtheorem{Def}{Definition}
\newtheorem{eg}{Example}
\newtheorem{re}{Reading}

\newtheorem{lemma}{Lemma}
\newtheorem{sublemma}{Lemma}[section]

\theoremstyle{remark}
\newtheorem{rmk}{Remark}

\newenvironment{solution}
  {\renewcommand\qedsymbol{$\blacksquare$}\begin{proof}[Solution]}
  {\end{proof}}

\begin{document}
\section*{Math 350 Introduction to Abstract algebra}
\subsection*{Problem set 1}

\begin{re}
	DF 1.6-1.7, 2.1-2.2
\end{re}
\begin{Prob} (Starred* problems are required)
	
	DF 1.2: 2, 3*, 7*
	
	DF 1.3: 1, 10*, 11*
	
	DF 1.4: 2, 4*, 5, 11abde*
\end{Prob}

\subsection*{Problem 1.2.3}
If some $x \in D_{2n}$ is not a power of $r$, by using (4) on page 25, we can rewrite $x = sr^i$ uniquely for some $i$. Now, note that $x^2 = sr^isr^i = (sr^is^{-1})r^i = (srs^{-1})^ir^i= (r^{-1})^ir^i= 1$. Since $x \neq 1$, it follows that $x$ has order 2. 

\subsection*{Problem 1.2.7}
Let $D = \langle a,b | a^2=b^2=(ab)^n=1\rangle$. Define $s=a$ and $r=a^{-1}b$. Then, observe that $s^2=a^2=1$. Furthermore, with order 2 element $a=a^{-1}$, $r^n=(a^{-1}b)^n = (ab)^n = 1$. Next, for $b$, we have that $rs = a^{-1}ba = aba = a(b^{-1}a) = ar^{-1} = sr^{-1}$ since $b=b^{-1}$. 
\subsection*{Problem 1.3.10}
Let us prove this with induction. Note that for the 'wrap around' cases, we implicitly define $a_0 := a_m$. For the base case, where $i=1$ for $\sigma = (a_1, a_2, ..., a_m)$, we have that $\sigma(a_k) = a_{k+1}$ mod $m$. For the inductive step, assume that the claim holds true for some $i$. To complete the proof, we solve for $i+1$. For $\sigma^{i+1} = \sigma(\sigma^i(a_k)) = \sigma(a_{k+i}) = a_{k+i+1}$. By induction, we have proved that $\sigma^i(a_k)=a_{k+i}$.

Next, note that $1 \leq i \leq m-1$, $\sigma^i(a_1) = a_{1+i mod m} \neq a_1$, which implies that $|\sigma| > m-1$. Furthermore, $\sigma^m(a_n) = a_{k+m mod m}$ for all $k \in {1,2,...,m}$, implying that $|\sigma| \leq m$. From these observations, we can deduce that $|\sigma| = m$.

\subsection*{Problem 1.3.11}
To prove the iff condition, we must prove the claim for both directions.

Suppose $\sigma^i$ is an $m$-cycle and FTSOC, assume that $gcd(i,m) = k > 1$. Then, by definition, there exists some $a,b \in \mathbb{N}$ such that $i=ad$ and $m=bd$. Then, we have that $(\sigma^i)^b = (\sigma^{ad})^b = (\sigma^m)^k = I$ since $\sigma$ is an $m$-cycle where $I$ is the identity permutation. Thus, we have that $|\sigma^m| \leq b < m$. This is a contradiction, since $\sigma^i$ is an $m$-cycle and thus $|\sigma^i| = m$. Thus, $i$ is coprime with $m$.

Note that from the above problem, we know that $\sigma^i = (1+i, 2+i, ..., m+i)$. Next, FTSOC, assume that $i,m$ are coprime, but that $\sigma^i$is not an $m$-cycle. This implies that there exists some $a,b \in {1,2,...,m}$ such that $a+i \equiv = b+i mod m$. Then, we have that $m | (b-a)$, which is a contradiction since $b-a \leq m$ by construction. Thus, we are done.

\subsection*{Problem 1.4.11}
\subsubsection*{a}
Evaluate $XY$ to obtain the following.
\begin{align*}
XY &= \begin{pmatrix}
1 & a & b\\
0 & 1 & c\\
0 & 0 & 1
\end{pmatrix} \cdot \begin{pmatrix}
1 & d & e\\
0 & 1 & f\\
0 & 0 & 1
\end{pmatrix} \\
&= 
\begin{pmatrix}
1 & d+a & e+af+b\\
0 & 1 & f+c\\
0 & 0 & 1
\end{pmatrix}
\\
\end{align*}
We can clearly see that $H(F)$ is closed under matrix multiplication by setting $a = a+d, b = b+e+af, c = f+e$ for generalized $X,Y$. To restrict a non-abelian condition, we can calculate $YX$ and compare and restrict terms. 
\begin{align*}
XY &= 
\begin{pmatrix}
1 & a+d & b+dc+e\\
0 & 1 & f+c\\
0 & 0 & 1
\end{pmatrix}
\\
\end{align*}
From comparing values, $H(F)$ will be non-abelian as long as $dc \neq af$.

\subsubsection*{b}

To find the inverse, let us first start with the determinant of $X$. By formula, we get the following

\begin{align*}
Det(X) &= Det \begin{pmatrix}
1 & a & b\\
0 & 1 & c\\
0 & 0 & 1
\end{pmatrix} \\
    &= 1 \cdot \begin{pmatrix}
    1 & c \\
    0 & 1
    \end{pmatrix}
    - a \cdot \begin{pmatrix}
    0 & c \\
    0 & 1
    \end{pmatrix}
    + b \cdot \begin{pmatrix}
    0 & 1 \\
    0 & 0
    \end{pmatrix} \\
    &= 1-a(0) + b(0) \\
    &= 1
\\
\end{align*}
Normally, this would be sufficient enough to prove the inverse property, as we know that the inverse exists from a nonzero determinant. However, finishing the calculations, our inverse $X^{-1}$ would be 
\begin{align*}
X^{-1} &= \begin{pmatrix}
1 & -a & -b+ac\\
0 & 1 & -c\\
0 & 0 & 1
\end{pmatrix}
\\
\end{align*}
Clearly this proves that $H(F)$ is closed under inverses as we can set $a = -a, b = b+ac, c = -c$ from the given formula of elements of $H(F)$, showing that $X^{-1} \in H(F)$.


\subsubsection*{d}
Because $F=\mathbb{Z}/2\mathbb{Z}$ only has two values, we can iterate across all possible values by casework. When $a=0,b=0,c=0$, this is the identity matrix, meaning that this element is order 1. When $a=0,b=1,c=0$, we have the case $X=x^{-1}$, meaning that this is an order 2 element. 

\subsubsection*{e}
Let $X \in H(\mathbb{R})$. Because $X \neq I$, we have that at least one of $a,b,c$ is nonzero. 

\begin{Prob}
	Let $G$ be a group $a_1,\dots a_r\in G$. We say that $a_1, \dots, a_r$ pairwise commute if $a_i$ and $a_j$ commute $\forall i,j$. We say $a_1,\dots, a_r$ are rank independent if $a_1^{e_1}\dots a_r^{e_r}=1$ implies that $e_i$ is a multiple of $|a_i|$ for all $i$. We want to prove the following Proposition:
	\begin{Prop}
		Let $G$ be a group and $a_1,\dots, a_r\in G$ be pairwise commuting and rank independent elements of finite order. Then $|a_1\dots a_r|=lcm(|a_1|,\dots, |a_r|)$.
	\end{Prop}
\begin{enumerate}
	\item Let $a,b\in G$ two commuting elements. Prove that $(ab)^n=a^nb^n$, $\forall n\in\mathbb{Z}$. (Hint: induction on $n$).
	\item If $a_1,\dots, a_r$ are pairwise commuting elements. Prove that $(a_1\dots a_r)^n=a_1^n\dots a_r^n$. (Hint: induction on $r$).
	\item Prove the proposition. (Hint: Use the definition of $lcm$ to check that the order is divisible by $lcm(|a_1|,\dots, |a_r|)$. Use that the order divides $lcm(|a_1|,\dots, |a_r|)$ by raising $a_1\dots a_r$ to the $lcm(|a_1|,\dots, |a_r|)$ power).
	\item Show that disjoint cycles in $S_n$ are pairwise commuting and rank independent. Deduce DF1.3 15.
\end{enumerate}
\end{Prob}

\begin{solution}
\begin{enumerate}
\item As the hint suggests, let's prove this using induction. For the base case, when $n = 0$, clearly $(ab)^0 = 1 = a^0b^0$. For the inductive step, first, assume that the claim holds for $k$. If $n \in \mathbb{Z}^+$, we have $(ab)^{k+1}=(a^kb^k)(ab)=a^{k+1}b^{k+1}$ from the associativity property and pairwise commutativity. Next, for $n \in \mathbb{Z}^-$, $(ab)^n = ((ab)^{|n|})^{-1} = b^{-|n|}a^{-|n|} = (b^{-1}...b^{-1})(a^{-1}...a^{-1})$. By the inverse property, it follows that $a^{-1},b^{-1}$ are also pairwise commutative. Thus, we can rearrange the product to get $(ab)^n = (a^{-1}...a^{-1})(b^{-1}...b^{-1}) = a^nb^n$, which completes the proof.

\item hello

\item To prove the equality, we will first prove two subclaims. We will first show that $|a_1...a_r|$ divides $lcm(|a_1|,|a_2|,...,|a_r|)$. Let $m = lcm(|a_1|,|a_2|,...,|a_r|)$. We can rewrite $m = |a_1|b_1=|a_2|b_2 = ... = |a_r|b_r$. Then, by the above claim, we have $(a_1...a_r)^m = a_1^m...a_r^m=(a_1^{|a_1|})^b_1...(a_r^{|a_r|})^b_r=e$. Thus, even if the lcm may not be the order of $a_1...a_r$, we still have that it divides $lcm(|a_1|,|a_2|,...,|a_r|)$.Next, we will show that $lcm(|a_1|,|a_2|,...,|a_r|)$ divides $|a_1...a_r|$. By definition, for some $k$, $(a_1...a_r)^k=e$. By the above claim, we can again rewrite $a_1^k...a_r^k=e$. Because $a_i$ are rank independent elements, it follows that $|a_i|$ divides $k$ for all $i$. By the definition of the lcm, it also follows that $lcm(|a_1|,|a_2|,...,|a_r|)$ divides $|a_1...a_r|$. Because these two terms divide each other, we can conclude that, in fact, $lcm(|a_1|,|a_2|,...,|a_r|) = |a_1...a_r|$.

\item 

To prove pairwise commutativity, WLOG, consider some pair of disjoint cycles $\alpha = (a_1, a_2,...,a_m)$ and $\beta = (b_1,...,b_n)$ as permutations of $S = {a_1,...,a_m,b_1,...,b_n,c_1,...,c_k}$. Since $\beta$ identity maps $a_i$ and vice versa for $\alpha$, we have that 
$$\alpha\beta(a_i) = \alpha(\beta(a_i)) = \alpha(a_i) = a_{i+1}$$
$$\beta\alpha(a_i) = \beta(\alpha(a_i)) = \beta(a_{i+1}) = a_{i+1}$$
So $\alpha\beta = \beta\alpha$ for $a_i$, and a similar argument can be made for $b_i$. Since $\alpha$ and $\beta$ both identity maps $c_i$, an even easier argument can be made for $\alpha\beta = \beta\alpha$ for $c_i$. Thus, we have proven pairwise commutativity.

Next, to prove rank independence. Suppose that $\sigma_1^{e_1}...\sigma_r^{e_r} = 1$ for cycles $\sigma_i$. Because the cycles are disjoint, we have that $\sigma_1^{e_1} = ... = sigma_r^{e_r} = 1$. As a result, we have that $|\sigma_1| \mid e_1,...,|\sigma_r| \mid e_r$, which is the definition of rank independence, so we are done.

*****ABOVE IS A LITTLE SKETCH, ESP WHERE WE JUMP TO IDENTITY FOR ALL ELEMENTS*****

Finally, since a permutation is defined as a product of cycles and we know from 1.4.10 that the order of the cycle is its length, with the above claim and (3), the claim that the order of an element in $S_n$ equals the lcm of the lengths of the cycles in its cycle decomposition easily follows.

Also, if $n$ is finite, then there's clearly finite order.
\end{enumerate}
\end{solution}

\begin{Prob}
	Show that $S_n$ is generated by each of the following set of permutations:
	\begin{enumerate}
		\item The set of transpositions $\{(j, j+1)|1\leq j<n\}$.
		\item The set of transpositions $\{(1,k)|1<k\leq n\}$.
		\item The set $\{(1,2),(1,2,\dots,n)\}$ (Hint: Reduce this to the case 1).
	\end{enumerate}
\end{Prob}

\begin{solution}
To proceed with the proof, we must first prove the following axiom

(1) 

(2)

(3) If we prove that the products of the permutations yields all transpositions of the form $(i,i+1)$, then by (1) the proof should follow.
\end{solution}

\footnotetext{Email address: pablo.boixedaalvarez@yale.edu}



\end{document}