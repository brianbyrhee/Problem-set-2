\documentclass[11 pt]{article}
\usepackage[english]{babel}
\usepackage{amsmath,amssymb,amsthm}

\title{Problem set 1}
\author{Pablo Boixeda Alvarez}

\newtheorem{Prob}{Problem}
\newtheorem{Prop}{Proposition}
\newtheorem{Opt}{Optional Problem}

\theoremstyle{definition}
\newtheorem{Not}{Notation}
\newtheorem{Def}{Definition}
\newtheorem{eg}{Example}
\newtheorem{re}{Reading}

\theoremstyle{remark}
\newtheorem{rmk}{Remark}

\newenvironment{solution}
  {\renewcommand\qedsymbol{$\blacksquare$}\begin{proof}[Solution]}
  {\end{proof}}

\begin{document}
\section*{Math 350 Introduction to Abstract algebra}
\subsection*{Problem set 1}

\begin{re}
	DF 1.6-1.7, 2.1-2.2
\end{re}
\begin{Prob} (Starred* problems are required)
	
	DF 1.2: 2, 3*, 7*
	
	DF 1.3: 1, 10*, 11*
	
	DF 1.4: 2, 4*, 5, 11abde*
\end{Prob}

\subsection*{Problem 1.2.3}
If some $x \in D_{2n}$ is not a power of $r$, by using (4) on page 25, we can rewrite $x = sr^i$ uniquely for some $i$. Now, note that $x^2 = sr^isr^i = (sr^is^{-1})r^i = (srs^{-1})^ir^i= (r^{-1})^ir^i= 1$. Since $x \neq 1$, it follows that $x$ has order 2. 

\subsection*{Problem 1.2.7}
Let $D = \langle a,b | a^2=b^2=(ab)^n=1\rangle$. Define $s=a$ and $r=a^{-1}b$. Then, observe that $s^2=a^2=1$. Furthermore, with order 2 element $a=a^{-1}$, $r^n=(a^{-1}b)^n = (ab)^n = 1$. Next, for $b$, we have that $rs = a^{-1}ba = aba = a(b^{-1}a) = ar^{-1} = sr^{-1}$ since $b=b^{-1}$. 
\subsection*{Problem 1.3.10}

\subsection*{Problem 1.3.11}

\subsection*{Problem 1.4.4}
Because $n$ is composite, by definition, there exists some $a$ where $1 < a < n$ that divides $n$. Note that it is impossible for some $b$ such that $ab \equiv 1 (\textrm{mod} n)$. FTSOC, assume that there exists some $b_0$ such that this statement holds true. Then, we can write the above congruence as $ab = 1 + kn = 1 + kam$. We can rewrite this as $a(b-km) = 1$

\subsection*{Problem 1.4.11}
\subsubsection*{a}
Evaluate $XY$ to obtain the following.
\begin{align*}
XY &= \begin{pmatrix}
1 & a & b\\
0 & 1 & c\\
0 & 0 & 1
\end{pmatrix} \cdot \begin{pmatrix}
1 & d & e\\
0 & 1 & f\\
0 & 0 & 1
\end{pmatrix} \\
&= 
\begin{pmatrix}
1 & d+a & e+af+b\\
0 & 1 & f+c\\
0 & 0 & 1
\end{pmatrix}
\\
\end{align*}
We can clearly see that $H(F)$ is closed under matrix multiplication by setting $a = a+d, b = b+e+af, c = f+e$ for generalized $X,Y$. To restrict a non-abelian condition, we can calculate $YX$ and compare and restrict terms. 
\begin{align*}
XY &= 
\begin{pmatrix}
1 & a+d & b+dc+e\\
0 & 1 & f+c\\
0 & 0 & 1
\end{pmatrix}
\\
\end{align*}
From comparing values, $H(F)$ will be non-abelian as long as $dc \neq af$.

\subsubsection*{b}

To find the inverse, let us first start with the determinant of $X$. By formula, we get the following

\begin{align*}
Det(X) &= Det \begin{pmatrix}
1 & a & b\\
0 & 1 & c\\
0 & 0 & 1
\end{pmatrix} \\
    &= 1 \cdot \begin{pmatrix}
    1 & c \\
    0 & 1
    \end{pmatrix}
    - a \cdot \begin{pmatrix}
    0 & c \\
    0 & 1
    \end{pmatrix}
    + b \cdot \begin{pmatrix}
    0 & 1 \\
    0 & 0
    \end{pmatrix} \\
    &= 1-a(0) + b(0) \\
    &= 1
\\
\end{align*}
Normally, this would be sufficient enough to prove the inverse property, as we know that the inverse exists from a nonzero determinant. However, finishing the calculations, our inverse $X^{-1}$ would be 
\begin{align*}
X^{-1} &= \begin{pmatrix}
1 & -a & -b+ac\\
0 & 1 & -c\\
0 & 0 & 1
\end{pmatrix}
\\
\end{align*}
Clearly this proves that $H(F)$ is closed under inverses as we can set $a = -a, b = b+ac, c = -c$ from the given formula of elements of $H(F)$, showing that $X^{-1} \in H(F)$.


\subsubsection*{d}
Because $F=\mathbb{Z}/2\mathbb{Z}$ only has two values, we can iterate across all possible values by casework. When $a=0,b=0,c=0$, this is the identity matrix, meaning that this element is order 1. When $a=0,b=1,c=0$, we have the case $X=x^{-1}$, meaning that this is an order 2 element. 

\subsubsection*{e}
Let $X \in H(\mathbb{R})$. Because $X \neq I$, we have that at least one of $a,b,c$ is nonzero. 

\begin{Prob}
	Let $G$ be a group $a_1,\dots a_r\in G$. We say that $a_1, \dots, a_r$ pairwise commute if $a_i$ and $a_j$ commute $\forall i,j$. We say $a_1,\dots, a_r$ are rank independent if $a_1^{e_1}\dots a_r^{e_r}=1$ implies that $e_i$ is a multiple of $|a_i|$ for all $i$. We want to prove the following Proposition:
	\begin{Prop}
		Let $G$ be a group and $a_1,\dots, a_r\in G$ be pairwise commuting and rank independent elements of finite order. Then $|a_1\dots a_r|=lcm(|a_1|,\dots, |a_r|)$.
	\end{Prop}
\begin{enumerate}
	\item Let $a,b\in G$ two commuting elements. Prove that $(ab)^n=a^nb^n$, $\forall n\in\mathbb{Z}$. (Hint: induction on $n$).
	\item If $a_1,\dots, a_r$ are pairwise commuting elements. Prove that $(a_1\dots a_r)^n=a_1^n\dots a_r^n$. (Hint: induction on $r$).
	\item Prove the proposition. (Hint: Use the definition of $lcm$ to check that the order is divisible by $lcm(|a_1|,\dots, |a_r|)$. Use that the order divides $lcm(|a_1|,\dots, |a_r|)$ by raising $a_1\dots a_r$ to the $lcm(|a_1|,\dots, |a_r|)$ power).
	\item Show that disjoint cycles in $S_n$ are pairwise commuting and rank independent. Deduce DF1.3 15.
\end{enumerate}
\end{Prob}

\begin{Prob}
	Show that $S_n$ is generated by each of the following set of permutations:
	\begin{enumerate}
		\item The set of transpositions $\{(j, j+1)|1\leq j<n\}$.
		\item The set of transpositions $\{(1,k)|1<k\leq n\}$.
		\item The set $\{(1,2),(1,2,\dots,n)\}$ (Hint: Reduce this to the case 1).
	\end{enumerate}
\end{Prob}

\begin{solution}
To proceed with the proof, we must first prove the following axiom

(1) 
(2)
(3) If we prove that the products of the permutations yields all transpositions of the form $(i,i+1)$, then by (1) the proof should follow.
\end{solution}

\footnotetext{Email address: pablo.boixedaalvarez@yale.edu}



\end{document}